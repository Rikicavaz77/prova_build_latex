\section{Riunione}
\subsection{Ordine del Giorno}
\begin{itemize}
	\item Incontro conoscitivo e scambio di informazioni generali tra i componenti del gruppo;
	\item Scelta del nome ufficiale del team;
	\item Raccolta di idee e suggerimenti per il logo;
	\item Creazione di un indirizzo email;
	\item Prima discussione del way of working;
	\item Scelta del responsabile del gruppo fino al 22/03/2024;
	\item Discussione iniziale dei capitolati, con relativi pro e contro;
	\item Programmazione della prossima riunione.
	
\end{itemize}

\subsection{Discussione e decisioni}

\subsubsection{Nome del team e logo}
Dopo aver chiarito gli impegni e la disponibilità dei singoli componenti, il gruppo ha vagliato la proposta di Sebastiano Lewental riguardante il nome da associare al team. Il termine “Argo” è stato accettato all’unanimità, in quanto richiama sia una parola utilizzata spesso in ambito informatico, sia un’espressione un po’ più ricercata e curiosa (considerando che Argo è il cane di Odisseo). Una volta scelto il nome, il team ha raccolto una serie di idee e suggerimenti sul design del logo.\\
Di seguito sono riportate alcune proposte:
\begin{itemize}
	\item Sagoma stilizzata di un cane in movimento;
	\item Raffigurazione dettagliata e moderna di un cane “futuristico”;
	\item Inserimento di elementi, come ad esempio delle barre orizzontali, che restituissero una sensazione di velocità e dinamismo.
\end{itemize}

Dopo aver generato alcuni prototipi con il supporto dell’Intelligenza Artificiale, il gruppo ha preferito realizzare un logo semplice e immediato. L’immagine è composta da una sagoma stilizzata e dal nome del team. Per restituire l’idea del movimento, il gruppo si è ispirato al dipinto a olio su tela di Giacomo Balla intitolato "Dinamismo di un cane al guinzaglio".

\subsubsection{Indirizzo e-mail}
Il gruppo ha deciso di usare Gmail come servizio di posta elettronica, poiché gratuito e facilmente accessibile. Il nome della casella di posta elettronica è composto da Argo e Unipd, così da mantenere un riferimento all’Università degli Studi di Padova. L’indirizzo email è stato creato da \martina{} e verrà utilizzato per tutte le comunicazioni tra fornitore e cliente o tra fornitore e committente.

\subsubsection{Way of working}
Il gruppo ha selezionato Telegram come chat di gruppo e Discord/Google Meet come piattaforma per svolgere le riunioni da remoto. Per quanto riguarda il sistema di versionamento, GitHub è stato considerato la scelta ottimale, in quanto già sperimentato in passato da diversi componenti del gruppo. In merito all’Issue tracking system, invece, il gruppo ha evidenziato due possibili alternative:
\begin{itemize}
	\item GitHub: la scelta più prudente e conservativa;
	\item Jira: un sistema di monitoraggio professionale e ricco di funzionalità, ma che richiede una curva di apprendimento notevole ed eventuali costi aggiuntivi.
\end{itemize}

Dopo una riflessione condivisa, il gruppo ha dato priorità alla facilità d'uso di GitHub ITS e alla sua integrazione con il sistema di versionamento adottato.
Relativamente alla stesura della documentazione di progetto, sono state valutate le seguenti opzioni:
\begin{itemize}
	\item Google Docs: garantisce una stesura semplice, immediata e collaborativa dei documenti, ma non si armonizza bene con il versionamento dei file;
	\item LaTeX: assicura una gestione e un controllo della documentazione efficaci, ma richiede tempo e studio per configurare correttamente l’ambiente.
\end{itemize}

Il gruppo ha deciso di utilizzare Google Docs per prendere appunti durante le riunioni e per compilare i documenti in attesa del template LaTeX. In seguito però, la documentazione di progetto verrà realizzata in LaTeX, non solo per il versionamento, ma anche per entrare in sintonia con uno strumento utile al percorso accademico.\\
Fino al 22/03/2024, il responsabile del gruppo Argo è \sebastiano.

\subsubsection{Discussione dei capitolati}
Ciascun membro del team ha esaminato ed esposto i lati positivi e negativi dei capitolati disponibili, rispettivamente C3, C6 e C9, sulla base delle presentazioni di metà ottobre. Essendo le idee poco chiare, si è deciso di fissare una riunione al giorno seguente, dando così modo ai componenti del gruppo di studiare e valutare i capitolati.

\clearpage

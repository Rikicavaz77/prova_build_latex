\section{Riunione}
\subsection{Ordine del Giorno}
\begin{itemize}
	\item Scelta definitiva del capitolato
\end{itemize}

\subsection{Discussione e decisioni}
\subsubsection{Discussione}
In seguito alla preferenza espressa dalla maggior parte del gruppo per il capitolato C6, sono stati contattati gli altri gruppi con l’obiettivo di avere un quadro generale delle preferenze e trovare un accordo comune. Di seguito sono riportati i risultati emersi dalle discussioni con gli altri gruppi.
\begin{itemize}
	\item Entrambi i gruppi esprimono come preferenza il capitolato C6;
	\item Negli altri gruppi è più netta la preferenza del capitolato rispetto a quella del gruppo Argo (57\% per C6, 43\% per C9).
\end{itemize}
Presentato il quadro generale si discutono le opzioni possibili per la candidatura:
\begin{itemize}
	\item Candidarsi a C6 concorrendo con gli altri gruppi;
	\item Candidarsi a C9.
\end{itemize}
Viene proposta inoltre una possibilità da attuare prima della candidatura:
\begin{itemize}
	\item Contattare la Proponente per C6 e il Committente per la proposta di espandere le disponibilità a 3 slot.
\end{itemize}
Il gruppo discute singolarmente le tre opzioni, analizzandone vantaggi e svantaggi.
\begin{outline}
	\1 \textbf{Candidatura a C6}
		\2 Vantaggi:
			\3 Scelta preferita dal gruppo in seguito all’analisi del documento di presentazione e alla riunione con la Proponente.
		\2 Svantaggi:
			\3 Competizione con altri gruppi che può causare il rifiuto della candidatura e il rinvio del kick-off.
	\1 \textbf{Candidatura a C9}
		\2 Vantaggi:
			\3 Rispetto degli slot disponibili per l’aggiudicazione degli appalti;
			\3 Minor numero di ostacoli nel flusso di lavoro rispetto alle altre opzioni. 
		\2 Svantaggi:
			\3 Indice di gradimento leggermente inferiore rispetto all'altro capitolo in esame.
	\1 \textbf{Avviare contatti per l'espansione a 3 slot di C6}
		\2 Vantaggi:
			\3 Permette la candidatura a C6 senza il rischio di essere respinti a causa di un numero eccessivo di candidature.
		\2 Svantaggi:
			\3 Non c’è garanzia di fattibilità;
			\3 Rallenta il processo di candidatura, in quanto non determina una scelta definitiva, avvicinando il gruppo alla scadenza.
\end{outline}
\subsubsection{Decisione}
La decisione definitiva viene posta a votazione, con risultato:
\begin{itemize}
	\item Candidatura a C6 (0 voti);
	\item Candidatura a C9 (5 voti);
	\item Contattare Proponente e Committente per allargamento a 3 slot (2 voti).
\end{itemize}
La scelta definitiva del capitolato per il gruppo è dunque C9: ChatSQL.
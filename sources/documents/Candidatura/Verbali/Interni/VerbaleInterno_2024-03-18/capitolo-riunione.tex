\section{Riunione}
\subsection{Ordine del Giorno}
\begin{itemize}
	\item Organizzazione per l’ultima revisione del documento di valutazione dei capitolati;
	\item Discussione sulla pianificazione del preventivo costi e lettera di presentazione;
	\item Valutazione di idee e proposte per migliorare il way of working.
\end{itemize}

\subsection{Discussione e decisioni}

All’inizio del meeting il gruppo si incontra per decidere quali membri si occuperanno dell’ultima revisione del documento di valutazione: questa prevede una sistemazione generale dei tre capitolati e una verifica degli aspetti positivi, negativi e delle conclusioni dei progetti SyncCity e ChatSQL, a seguito del cambio del capitolato scelto come proposta di candidatura. \\
In seguito, il focus della riunione si sposta sugli ultimi documenti da redigere per la candidatura: 
\begin{itemize}
	\item Il documento di preventivo dei costi, che raccoglie le informazioni sul costo finale, la data di consegna, gli impegni e le possibili difficoltà riscontrabili durante il progetto;
	\item La lettera di presentazione del gruppo.
\end{itemize}
Infine si espongono gli aggiornamenti e i possibili miglioramenti sull’impostazione del way of working del gruppo. Questi prevedono:
\begin{itemize}
	\item Utilizzo di Google Docs per le versioni non ufficiali dei documenti, così da permetterne una stesura più rapida;
	\item Sistema di versionamento dei documenti;
	\item Definizione delle versioni ufficiali dei documenti in LateX;
	\item Utilizzo dei branch di GitHub per il versionamento, con la decisione di utilizzare i branch principali solo per il rilascio di funzionalità definitive;
	\item Divisioni più concrete per la redazione, la verifica e l’approvazione dei documenti.
\end{itemize}

\subsubsection{Obiettivi fissati}
\begin{itemize}
	\item Divisione in sottogruppi per la stesura del preventivo costi, la revisione finale del documento di valutazione dei capitolati e lettera di presentazione;
	\item Redazione in LateX dei verbali esterni per la firma da parte delle aziende con cui è stata organizzata una riunione;
	\item Verifica e approvazione dei verbali interni.
\end{itemize}
I gruppi per i documenti da redarre sono:
\begin{itemize}
	\item Preventivo costi
	\begin{itemize}
		\item \mattia
		\item \marco
		\item \sebastiano
		\item \riccardo
	\end{itemize}
	\item Documento di valutazione capitolati
	\begin{itemize}
		\item \raul
		\item \martina
	\end{itemize}
	\item Lettera di presentazione
	\begin{itemize}
		\item \martina
	\end{itemize}
	\item Verbali interni
	\begin{itemize}
		\item \raul
		\item \tommaso
	\end{itemize}
\end{itemize}